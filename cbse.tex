\documentclass[12pt]{article}
\usepackage{hyperref}
\usepackage{listings}
\usepackage{biblatex}
\usepackage{tikz}
\usepackage{refstyle}
\usepackage{mathabx}
\usepackage{amssymb}
\usepackage{caption}
\usepackage{float}
\usepackage{graphicx}
\usepackage{graphics}
\usepackage{subfig}
\usepackage{tfrupee}
\usepackage{tabularx}
\usepackage{amsmath} 
\graphicspath{{/storage/self/primary/Download/latexnew/fig}}
\graphicspath{{figs/}}\providecommand{\mydet}[1]{\ensuremath{\begin{vmatrix}#1\end{vmatrix}}}
\providecommand{\myvec}[1]{\ensuremath{\begin{bmatrix}#1\end{bmatrix}}}\providecommand{\cbrak}[1]{\ensuremath{\left\{#1\right\}}}
\providecommand{\brak}[1]{\ensuremath{\left(#1\right)}}\graphicspath{{/storage/self/primary/Download/latexnew/fig}}
\title{12th cbse}
%\author{Prof.G V V sharma}
\date{\today}
\begin{document}
\maketitle{Questions}
\begin{enumerate}

\item If $f(x)=(\frac{1-x}{1+x})$,then find $(f\circ f)(x)$.

\item Let W denote the set of words in the English dictionary. Define the relation R by

$R = \{(x, y) \in W \times W \text{such that} \text{ x  and y }  \text{ have at least one letter in common}\}$.
Show that this relation R is reflexive and symmetric, but not transitive.

\item Find the inverse of the function $f(x) = \frac{4x}{3x+4}$.

\item $\int x \sqrt{x + 2}$ dx is equal to

(A) $\frac{2}{5}(x + 2)^{\frac{5}{2}}$ - $\frac{2}{3}(x + 2)^{\frac{3}{2}}$ + C

(B)$\frac{5}{2}(x + 2)^{\frac{5}{2}}$ + $\frac{3}{2}(x + 2)^{\frac{3}{2}}$ + C

(C)$\frac{2}{5}(x + 2)^{\frac{5}{2}}$ - $\frac{4}{3}(x + 2)^{\frac{3}{2}}$ + C

(D)$\frac{2}{5}(x + 2)^{\frac{5}{2}}$ + $\frac{4}{3}(x + 2)^{\frac{3}{2}}$ + C


where \(C\) is the constant of integration.
\end{enumerate}
\end{document}
